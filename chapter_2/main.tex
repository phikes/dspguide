\documentclass{article}
\usepackage[utf8]{inputenc}

\title{Chapter 2 - Mean \& Standard Deviation}
\author{Phillip Kessels}
\date{April 2020}

\begin{document}

\maketitle

\section*{Exercise 1}
\begin{itemize}
    \item Write three functions in Octave which calculate the \textbf{mean, average deviation and standard deviation} \textit{respectively} in their closed forms from equation 2-2 and 2-1 for $N$ samples.
    \item Write a function in Octave which calculates the \textbf{standard deviation suitable for running statistics}. This should have the number of samples, sum of samples, sum of the squares of the samples and a new list of samples as input.
    \item Calculate the mean, average deviation and standard deviation in both forms for the samples: $$(x_1,x_2,x_3,x_4,x_5,x_6,x_7,x_8,x_9,x_{10})=(2,4,4,4,5,5,7,9)$$ \textit{Hint: $\mu = 5$, $\sigma = 1.5$, $\sigma^2\approx4.5714$}
    \item Calculate the signal-to-noise ratio, coefficient of variation and typical error for the above samples under the assumption that the mean is the desired signal. Describe what these numbers mean.
    \item Plot a sine wave of 1 Hz with unit amplitude for 10 seconds. In the same graph plot its average deviation, its standard deviation using your functions from above, its root-mean-square and derived from the root-means-square its peak-to-peak amplitude.
    \item Derive the formula for the standard deviation for running statistics.
\end{itemize}

\end{document}
